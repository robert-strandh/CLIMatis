\chapter{Changes from previous version}

\section{General}

This section explains the general changes that we consider for \clim{}
compared to \climtwo{}.  Following sections will contain specific changes
with respect to particular features of \climtwo{}.

We intend to act upon one of the ``major issues'' indicated in the
\climtwo{} specification, namely a suggested merge of the \emph{sheet} and
\emph{output record} concepts.  This will greatly simplify the design
of \clim{}.

There are many issues with the \climtwo{} specification.  For
instance, what happens to an output record (which is rectangular
according to the specification) when it is in a pane that has an
arbitrary sheet transformation that does not preserve this shape?  For
\clim{} we want to clarify what happens in such situations, and
eliminate it as a part of the specification if the concept is hard or
impossible to implement. 

Several aspects of the \climtwo{} specification make it hard to
localize applications.  One such aspect is the fact that command-line
names are part of the command table, requiring a different command
table for each language.  We want to clean up such aspects so as to
prepare \clim{} for the possibility of writing applications using
different languages. 

By clarifying the semantics of \emph{zones} (which replace \climtwo{}
\emph{output records}), especially with respect to the exact time when
the effect of manipulating the hierarchy of these objects is visible,
we avoid the almost inevitably quadratic algorithms involved in
manipulating \climtwo{} output records. 

We keep the layout protocol, but clarify when it is invoked.  We
replace space requirements by vertical and horizontal elasticity
functions. 

We eliminate mirrored sheets as an explicit abstraction of \clim{} and
let each backend decide on a case-by-case basis whether a zones should
have a mirror or not.

We eliminate alignment and spacing options to the layout panes and
instead provide a zone type that is very rigid (for spacing) and a
zone type that is very elastic (for alignment).

\section{Regions and Transformations}

In \climtwo{} this is a very general concept.  We will probably keep
these, but the \clim{} equivalent of a \climtwo{} sheet (called a
\emph{zone}) will not admit arbitrary regions and transformations. 

\section{Sheets}

\section{Commands and command tables}

\section{Internationalization}

Define a classes for country, language, printing of numbers, units,
etc, and locale classes that inherit from subclasses of those.

\section{Text styles}

We keep the concept of text styles, but we may remove the concept of
merging text styles.  We add the possibility of naming font families. 
We will specify that text-style-ascent, text-style-descent,
text-style-height, and text-style-width may only be called as part of
the layout protocol, because they are only defined when the zone is
attached to a port. 

\section{Layout protocol}

We guarantee that the layout protocol will only be run when the zone
hierarchy is attached to a port.  Thus, when a text zone is asked to
provide space requirements, then it is safe for it to ask the size of
the text, because the text style then has a mapping to a device font. 

