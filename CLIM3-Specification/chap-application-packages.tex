\chapter{Packages}

\clim{} contains several packages that are relevant to the application
writer.  In addition it contains a large number of internal packages
that the application writer does not need to be aware of.

The main packages that the application writer would typically use are:

\begin{itemize}
\item \code{clim3}.  This package contains most of the symbols that
  are used for basic functionality such as the names of standard
  zones, keyboard handlers, button handlers, etc. 
\item \code{clim3-gadgets}.  This package contains names of convenient
  components for use in graphic user interfaces such as buttons,
  butcons, scrollbars, menus, etc.
\end{itemize}

We strongly recommend that the application writer avoid the option
\code{:use} of \code{defpackage} or the function \code{use-package}
with any of these packages as argument, and instead recommend using
an explicit package prefix whenever a symbol from any of those
packages is required.

Since \clim{} uses a stratified design, it is occasionally useful for
the application writer to write application-specific extensions to
\clim{}.  In such situations, the application writer will often use
the package \code{clim3-ext}.  Guidelines for using that package are
provided in the part named ``Extension writer's guide''. 

