\chapter{Future development}

\section{Short-term improvements}

This section contains a 'to-do' list of short-term possible
improvements.  The feasibility of some of these improvements has not
been established, so they might not be possible or practical. 

\begin{itemize}
\item Document that \code{vbrick} is a special case of \code{vframe}
  and \code{hbrick} is a  special case of \code{hframe} and check
  beforehand that this is actually true. 

\item The setters for the height and the width of a zone should
  probably be in the \code{clim-ext} package.  Client code that needs
  to change the dimensions of a zone should set the sprawls instead. 

\item Currently we have \code{impose-size}, \code{set-vpos},
  \code{natural-size}.  Add functions \code{(setf clim-ext:width)}
  \code{(setf clim-ext:height)}, \code{natural-width},
  \code{natural-height}.

\item Investigate the possibility of using the MOP to verify that when
  a zone class is finalized, it has a good selection of superclasses,
  so that it reacts to the notification protocols in appropriate ways.

\item The function \code{(setf clim3:children)} should probably not be
  available to all compound zones. Each zone type should be
  responsible for how its children are assigned, and for setting the
  \code{parent} of its children.  The function \code{(setf
    clim3:children)} should be seen as a convenience for compound
  zones with a reasonably small number of children, and for these
  zones it provides functionality for setting the \code{parent} of
  those children.  The question, then, is what we mean by
  \code{compound-zone}.  It could either mean any zone that can have
  children, or it can mean the particular type of zone that also
  supports \code{(setf clim3:children)}.  Either way, we need to come
  up with a new name. 

\end{itemize}
