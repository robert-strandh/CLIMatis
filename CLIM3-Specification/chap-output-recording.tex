\chapter{Output recording}

\section{Introduction}

Applications typically maintain some kind of data structure that is
modified as a result of executing a \clim{} \emph{command object}.  To
generate visible output from such a data structure, the application
will have to produce a hierarchy of zones that \clim{} can draw on
the canvas.  \clim{} facilitates this task by letting the application
programmer use seemingly low-level primitives such as \code{draw-line}
and \code{draw-text} and automatically converting them to zones.
Also, \clim{} maintains a \emph{cursor} indicating the position of
output within the parent zone, making it possible for the
application to use ordinary \cl{} output primitives such as
\code{print} or \code{format}.  Each line of text is automatically
converted to a zone. 

A simple application such as an IRC client or some other application
that generates an increasing amount of output at each iteration of the
command loop, might indicate to \clim{} that the hierarchy of zones
of some pane should not be cleared each time around the command loop,
and that each piece of output should simply be added to the
hierarchy.  Such an application might occasionally explicitly request
that the hierarchy be cleared, but \clim{} will never do it
automatically. 

\section{Incremental redisplay}

Applications can be written with various degrees of performance in
mind.  A very basic application may simply clear the application pane
and traverse the entire application data structure at the end of each
iteration of the command loop and generate fresh output for its entire
data structure.  For instance, a text-editor application might have a
data structure consisting of paragraphs, lines, words, and characters.
Such an application might scan the entire data structure and generate
a hierarchy of zones corresponding to its internal data structure. 

A simple application such as the one above might be improved in terms
of performance by taking advantage of the possibility of \clim{} to
compare two hierarchies of zones, one at the beginning of an
iteration of the command loop, and the other one at the end.  By
a slight modification to the application logic, one might insert a
test for each zone to be generated that will verify if the contents
of the zone is unchanged since the previous iteration of the command
loop.  If this is the case, the entire generation of the output can be
omitted and \clim{} can reuse the existing zone instead.  The
advantage of this technique is that it requires only minor
modifications of the very simple application logic that consists of
scanning the entire data structure each time around the command loop,
and that performance can be quite acceptable, even for fairly large
data structures.  Imagine again the text-editor application above.  In
most cases, a single line in a single paragraph has changed.  The
zones generated by most paragraphs can thus be reused.  While each
paragraph has to be tested, most tests will result in no further
traversal of the paragraph being necessary.  The time spent by the
application is thus linear in the number of top-level data-structure
elements (paragraphs) in most cases.  

A more compound application will manipulate the hierarchy of zones
explicitly.  For instance, a word-processor application might have a
data structure consisting of chapters, sections, paragraphs, words,
and characters.  A complicated incremental page-layout algorithm might
convert this data structure to another one, consisting of pages,
paragraphs, lines, and words.  For performance reasons, the
page-layout algorithm needs to be incremental, so the application
needs to maintain a copy of the resulting data structure and only
update it when necessary.  An application of this degree of complexity
would simply map each element of the resulting data structure (pages,
paragraphs, lines, words) to a \clim{} zone.  At each iteration of
the command loop, such an application can still take care of the
capability of \clim{} to compare two data structures, by generating a
new hierarchy each time around the command loop.  Most of the
top-level zones (corresponding to pages) will be the same, and
within the pages that have changed, most paragraphs will be the same,
etc.  \clim{} is thus capable of minimizing the update of the screen at
a cost that most of the time corresponds to iterating over each
top-level zone. 

