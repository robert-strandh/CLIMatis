\chapter{Contributing to \sysname{}}

\section{Coding style}

\subsection{Commenting}

Comments are meant to be read by the maintainers of the code.  One can
therefore safely assume that the reader is quite familiar with the
\cl{} language and the main structure of \clim{} and \sysname{}.

Comments should be used to explain aspects of the code that are not
obvious from reading the code itself.  A comment is the ideal place to
introduce definitions of concepts that must be understood in order for
the code to make sense.  If the code is structured in a particular way
for performance reasons, then a comment is a good place to indicate
such a fact, so as to avoid that the maintainer be tempted to
introduce modifications that alter this structure. 

\subsection{Designators for symbol names}

Always use uninterned symbols (such as \texttt{\#:hello}) whenever a
string designator for a symbol name is called for.  In particular,
this is useful in \texttt{defpackage} and \texttt{in-package} forms.

Using the upper-case equivalent string makes the code break whenever
the reader is case-sensitive (and it looks strange that the designator
has a different case from the way symbol that it designates is then
used), and using keywords unnecessarily clutters the keyword package.

\subsection{Docstrings}

We believe that it is a bad idea for an implementation of a Lisp
system to have docstrings in the same place as the definition of the
language item that is documented, for several reasons.  First, to the
person reading the code, the docstring is most often noise, because it
is known from the standard what the language item is about.  Second,
it often looks ugly with multiple lines starting in column 1 of the
source file, and this fact often discourages the programmer from
providing good docstring.  Third, it makes internationalization
harder.

For this reason, we will provide language-specific files containing
all docstrings of Common Lisp in the form of calls to \texttt{(setf
documentation)}. 

\subsection{Naming and use of slots}

In order to make the code as safe as possible, we typically do not
want to export the name of a slot, whereas frequently, the reader or
the accessor of that slot should be exported.  This restriction
implies that a slot and its corresponding reader or accessor cannot
have the same name.  Several solutions exist to this problem.  The one
we are using for \sysname{} is to have slot names start with the
percent character (`\texttt{\%}').  Traditionally, a percent character
has been used to indicate some kind of danger, i.e. that the
programmer should be very careful before directly using such a name.
Client code that attempts to use such a slot would have to write
\texttt{package::\%name} which contains two indicators of danger,
namely the double colon package marker and the percent character.

Code should refer to slot names directly as little as possible.  Even
code that is private to a package should use an internal protocol in
the form of readers and accessors, and such protocols should be
documented and exported whenever reasonable.  It sometimes good
practice to have multiple accessors for a slot, one for internal
purposes and one for use by client code.  This practice allows for
\code{:before}, \code{:after}, and \code{:around} methods on one
accessor but not the other. 

\subsection{Using other packages}

The \code{:use} option of \code{defpackage} and the \code{use-package}
function should be restricted to the \code{common-lisp} package as
much as possible.  The reason for this restriction is that using a
package this way represents a commitment to accepting all exported
symbols of that package, current and future, whereas in most cases
there is no guarantee that future modifications of the package will
not introduce symbol conflicts.  If it is desired to avoid explicit
package prefixes in some cases, then it is better to use the
\code{:import-from} option of \code{defpackage} to import an
explicitly-supplied list of symbols.

\subsection{Conditions, restarts, and reporting}

Conditions should not be simple conditions, because we want condition
reporting to be subject to internationalization.   For the same
reason, the condition reporter should not be part of the
\code{define-condition} form, and instead be written separately in a
file that contains language-specific condition reporters. 

Restarts should be provided whenever practical. 

\subsection{Internationalization}

We would like for {\sysname} to have the ability to report messages in
the local language if desired.  The way we would like to do that is to
have it report conditions according to a \texttt{language} object.
The details of this mechanism will be determined later.
\subsection{Threading and thread safety}

Consider the use of locks to be free.  A technique called
``speculative lock elision'' is already available in some processors,
and we predict it will soon be available in all main processors.

