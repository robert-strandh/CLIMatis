\chapter{Optimal redrawing}

\section{Introduction}

The purpose of optimal redrawing is not mainly to touch as few pixels
as possible in the visible area of the screen, even though it can do
that as well.  Instead, the purpose is to simplify applications by
having \clim{} manage a very large number of zones, most of which
are invisible at any point in time.  Imagine for instance a word
processor or an editor for music scores.  Such an application might
produce hundreds of pages of output.  Instead of leaving the burden to
the application writer to determine what part is visible, and how
large the scroll bar should be, it would be good if the application
can produce the entire output, and let \clim{} deal with scrolling and
incremental modifications to the output.  

Such an application might be written to produce a hierarchy of
zones.  At a high level, each page might be a zone.  Each such
page can contain paragraphs or, in the case of a score editor,
\emph{systems}, then lines, words, and perhaps individual characters.
This output is visible through a small viewport that might cover a
single page, or at most a few pages.  The main purpose of the
algorithm presented here is to avoid invoking code to draw parts of
the output that are currently invisible and parts that have not changed
since the previous time.  For instance, after a single modification of
a page of text, most paragraphs might be intact.  The algorithm
described here is able to avoid redrawing such paragraphs. 

In other words, the application writer does not explicitly manage
visibility, and can write the application as if every part of the
output is visible.

The algorithm makes two passes over a hierarchy of zones, the first
one from top to bottom, and the second one from bottom to top.  For
this reason the sequence of zones is probably best represented as a
doubly-linked list.

In the first pass, a region to be redrawn is computed.  This region
starts out as empty and is potentially added to for each zone in a
sequence.  

In the following algorithm, the region is clipped by zones that are
not visible after the redraw.  A zone can be in one of six states:

\begin{itemize}
\item Unchanged.  This means that the zone has already been drawn in
  the past, and that it hasn't been moved or altered since the last
  time it was drawn.
\item Deleted.  This means that the zone has already been drawn in
  the past, but is no longer supposed to be.  A deleted zone
  might also be marked as moved and/or modified, but ``Deleted'' takes
  precedence.  The region that is needed by the redraw algorithm is
  the region the zone had last time it was drawn, i.e., before
  it was moved or modified.
\item Inserted.  This means that the zone has not been drawn in the
  past, but is supposed to be now.  An inserted zone might also be
  marked as moved and/or modified, but ``Inserted'' takes precedence.
  The region that is needed by the redraw algorithm is the region that
  the zone has after having been moved or modified.
\item Moved and not modified.  This means that the zone has already
  been drawn in the past, but that it has simply changed positions
  since the last time it was drawn.
\item Modified but not moved.  Only a compound zone can be marked as
  modified, indicating that a child zone has been
  inserted, deleted, moved, or modified.
\item Moved and modified.  In this case the zone is treated as
  deleted from its position when last drawn, and inserted at its
  final position. 
\end{itemize}

Applications are responsible for:

\begin{itemize}
\item Marking a zone as ``Modified'' when any of its children is
  either inserted, deleted, moved, or modified.
\item Marking a zone as ``Moved'', whenever it is moved. 
\end{itemize}

The \clim{} redraw algorithm provides the following services, so the
application need not do so:

\begin{itemize}
\item Marking a newly created zone as ``Inserted''. 
\item Automatically detecting that a zone has been deleted, and
  marking it as ``Deleted''. 
\end{itemize}

Only newly created zones can be inserted into a higher-level
zone.  In particular, applications are not allowed to change the
stacking order of zones. 

In the first pass, the redraw algorithm computes the region to be
redrawn.  This is done from the top zone to the bottom zone.  In
each iteration, it modifies the region to be drawn and passes it on to
the next iteration, according to the following cases:

\begin{itemize}
\item If a zone is unchanged, then pass on the region without any
  modification. 
\item If a zone has been deleted, then add its area (i.e, the area
  it occupied when it was last drawn) to the region.
\item If a zone is new (has been inserted since the last redraw),
  then if it is covered by some other visible zone, then add the
  area that is covered to the region.  This is an important
  optimization, because when a new zone is drawn on top, what is
  underneath does not need to be redrawn. 
\item If a zone has been moved (whether also modified or not), then
  in the most general case, treat it as one deleted and one inserted
  zone.
\item If a zone has been modified but not moved, recursively
  traverse the children.
\end{itemize}

In the second pass, the region that has been computed in the first
pass is used to draw the zones from bottom to top.

\begin{itemize}
\item If a zone does not overlap the region, then do nothing
\item If it does overlap the region, and it is an atomic zone,
  request an \emph{redisplay} of the relevant region from the
  application. 
\item If it does overlap the region and it is a compound zone,
  recursively traverse its children.
\end{itemize}

\section{Details of algorithm}

The algorithm for optimal redrawing is divided into two parts.  The
first part computes the region to be redrawn, and the second part
redraws that region.

To compute the region to redraw, the algorithm manages three regions:

\begin{itemize}
\item A region called \texttt{region-to-redraw}, which is the region
  that ultimately becomes the final result of the algorithm.  This
  region grows as the algorithm traverses zones from top to bottom.
  The initial value for this region is \texttt{+nowhere+}.
\item A region called \texttt{visible-region-after} which is the
  region that is visible after the redraw.  This region starts off as
  \texttt{+everywhere+} and may decrease in size as a result of opaque
  zones.
\item A region called
  \texttt{visible-translucent-covering-region-after}, which is a
  region containing potentially translucent zones \emph{after}
  redraw.  This region allows for an important optimization in that
  when a region needs to be redrawn and it is not covered by any
  translucent zones, it can be drawn on top of what is already
  there.  If, however, it has translucent zones on top, the region
  needs to be ``erases'', meaning all zones in that region need to
  be redrawn from bottom to top. 
\end{itemize}

The algorithm to compute the region to redraw thus takes four
arguments: a zone and the three regions indicated above.  It
traverses the zones from top to bottom, and in each step, it
modifies the three regions, and passes on the modified regions to the
next iteration.  A complete description of the algorithm must specify
for each type of zone how the regions are modified. 

For an \emph{opaque} zone, the regions are modified as follows:

\begin{itemize}
\item \texttt{region-to-redraw}.  If the zone has been deleted, then
  add its visible subregion to \texttt{region-to-redraw}.  Otherwise,
  this region stays the same.
\item \texttt{visible-region-after}.  Of the zone has been inserted,
  or is the same, from this region we subtract the region of the
  zone, because of the opacity of the zone.
\item \texttt{visible-translucent-covering-region-after}.  This region
  is treated the same way as the previous one.
\end{itemize}

For a \emph{atomic translucent zone}, the regions are modified as
follows:

\begin{itemize}
\item \texttt{region-to-redraw}.  This is an atomic zone, so it is
  either inserted, deleted, or unchanged.  If it is unchanged,
  this region is not modified.  If it is deleted, the
  visible part of the region of the zone is added to this region.
  If it is inserted, the visible part of the region of the zone that
  is also covered by translucent zones.  The part that is not
  covered by translucent zones can be drawn on top of existing
  pixels without the need to redraw.  Thus,
  we first compute the intersection of the inserted region of the zone
  with \texttt{visible-translucent-covering-region-after}.
\item \texttt{visible-region-after}.  This region does not change for
  translucent zones. 
\item \texttt{visible-translucent-covering-region-after}.  If the
  zone has been inserted or is unchanged, then the visible
  part of the zone is added to this region.  This region does not
  change for a deleted zone.
\end{itemize}

Finally, the complicated case is for a \emph{compound translucent
  zone}.  We would like to avoid scanning the children if possible,
  at the risk of having more zones redrawn at the end.  Individual
  zones will indicate that a minor change has been made to its
  children by marking it modified and that a major change has been
  made by replacing the old zone by a new one, in effect deleting
  the old one and inserting a new one.  We consider a compound zone
  to be relatively densely covered by translucent children, so that
  the entire region of the zone can be passed on as being a
  translucent covering region.

\begin{itemize}
\item \texttt{region-to-redraw}.  If the zone is unchanged, we pass
  this region on unchanged.  If the zone has been inserted or
  deleted, we treat this region in the same way as for an atomic
  translucent zone.  Finally, if zone has been modified, we
  recursively consider the children of this zone.  The initial value
  of this region for the children is the region of the zone after
  redraw intersected with the union of the region of the zone
  before and after redraw.  The result is added to
  \texttt{region-to-redraw} and passed on.
\item \texttt{visible-region-after}.  This region is always passed on
  unchanged as with translucent atomic zones. 
\item \texttt{visible-translucent-covering-region-after}.  Unless the
  zone has been deleted, the visible part of the zone is added to
  this region.
\end{itemize}

