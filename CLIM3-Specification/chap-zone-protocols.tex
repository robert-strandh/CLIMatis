\chapter{Zone protocols}

\section{Purpose}

The purpose of the zone protocols is twofold:

\begin{itemize}
\item The main purpose is to make certain operations simple and
  transparent to applications.  See below for details.
\item The secondary purpose is to optimize rendering so that even for
  very large data sets, the response time is reasonable.
\end{itemize}

Applications should be able to manipulate both connected and
disconnected zone hierarchies in a painless way.  Operations available
to an application are:

\begin{itemize}
\item An application should be able to move a zone by setting its
  horizontal and vertical positions, provided it is not the child of a
  layout zone that specifically disallows this operation.  The result
  of the operation should be immediately visible after the next
  iteration of the event loop.
\item An application should be able to modify the depth of a zone,
  provided it is not the child of a layout zone that specifically
  disallows this operation.  Again, the result of the operation should
  be immediately visible after the next iteration of the command
  loop. 
\item An application should be able to change the children of a
  compound zone.  This is the main operation used to manipulate the
  zone hierarchy.  
\end{itemize}

Notice that the effects of the modifications are synchronized with the
event loop, and that the event loop is responsible for the ultimate
rendering of the zone hierarchy.  Therefore, an application can make
multiple changes to the zone hierarchy as a result of a single event,
and only the final result of those changes will be rendered.  

Let us examine the consequences of these operations in greater
detail. 

\subsection{Moving a zone}

If the zone is the root zone, then we might want to ask the display
server to move the corresponding window. 

If the zone is the child of bboard zone, then we need to render it in
a different place next time. 

If the zone is the child of a scroll zone, the scrollbar will have
to be rendered differently next time. 

If the zone if part of a layout zone that imposes positions of the
child, then an error will be signaled. 

\subsection{Modifying the depth of a zone}

If the zone is a root zone, we might ask the window manager to raise
or bury the corresponding window, but we might do nothing. 

Otherwise, the sibling zones will be rendered in a different order
next time.

\subsection{Changing the children of a zone}

If the zone is a bboard zone, new children must have valid sprawls so
that we can compute their natural sizes before rendering

If the zone is a scroll zone, we must make sure the child has a
valid sprawl so that we can compute its natural size before rendering.

If the zone is the child of a layout zone that computes its sprawls from
the sprawls of its children, then the parent must have its sprawls
invalidated, so that they are recomputed next time it is rendered.
This invalidation must propagate up the hierarchy until a zone is
reached that does not take child sprawls into account. 

\section{Zone geometry protocol}

\subsection{Sprawl}
\label{sec-zone-protocols-geometry-sprawl}

The desired dimensions of a zone are expressed in the form of objects
of type \emph{sprawl}.  A sprawl is an object with three components:
the \emph{minimum size}, the \emph{preferred size} and the \emph{maximum
  size}.  Each zone has one horizontal sprawl and one vertical sprawl.

The three components of a sprawl are interpreted as \emph{desired}, so
that the minimum size is really the \emph{desired minimum size}, the
preferred size is really the \emph{desired preferred size}, and the
maximum size is really the \emph{desired maximum size}.  Every zone
must be prepared to take on any size that is imposed on it by the
parent zone or by the display server.  \clim{} will try as much as
possible to honor the desired sizes, but sometimes it is not possible,
so some other size will be chosen.

The minimum and preferred sizes of a sprawl are non-negative
integers.  The maximum size is either a non-negative integer, or
\texttt{NIL}, indicating that the zone has no desired maximum size. 
The preferred size is greater than or equal to the minimum size and
the maximum size is either \texttt{NIL} or it is greater than or equal
to the preferred size. 

\subsubsection{Combining sprawls}

Sprawls can be combined either in \emph{series} (like the vertical
sprawls of the zones inside a vbox), or in \emph{parallel} (like the
horizontal sprawls of a the zones inside a vbox).  

When combined in series, the different components are then summed up
individually so that the minimum size is the sum of each individual
minimum size, the preferred size is the sum of each individual
preferred size, and the maximum size is the sum of each individual
maximum size, or \texttt{NIL} if any individual maximum size is
\texttt{NIL}.

Combining sprawls in parallel is a more difficult operation.  The
basic idea is that the minimum size is computed as the maximum of each
individual minimum size, the preferred size is computed as the average
of the preferred sizes, and the maximum size is computed as the
minimum of each individual maximum size, or \texttt{NIL} if all of the
individual maximum sizes are \texttt{NIL}.  However, there are many
situations where this method is not possible, perhaps because the
minimum size of one individual sprawl is greater than the maximum size
of another.  In that case, some compromise is made, the details of
which is not terribly important.  Applications should avoid situations
like that though. 

\subsubsection{Distributing space among combined sprawls}

When some parent zone imposes the size of child zones whose sprawls
have been combined in series, the imposed size of the parent must be
distributed among the individual child zones.  In some situations,
this is easy, such as when the imposed size of the parent is greater
than the sum of the preferred sizes of its children, and some maximum
child size is \texttt{NIL}.  In that case, the difference between the
imposed size of the parent and the sum of the preferred sizes of the
children is distributed evenly among the children having a
\texttt{NIL} maximum size.

FIXME: say more....

When some parent zone imposes the size of child zones whose sprawls
have been combined in parallel, the same size is imposed on each child
zone.

\subsection{Generic functions}

\subsubsection{Generic function \texttt{hpos}}

Return the current horizontal position of the zone relative to its
parent.  

If the zone is the root zone and it is connected to some display
server, then depending on the display server, this value may or
may not reflect the position of the corresponding top-level
window on the display. 

\subsubsection{Generic function \texttt{(setf hpos)}}

Set the current horizontal position of the zone relative to its
parent.  

Calling this function is the normal way of setting the position of
a zone that is the child of a layout zone that does not impose the
position of its children, such as a bboard zone or a scroll
zone.

Setting the position of a zone may not have the effect that an
application expects.  If the zone is the child of a layout zone
that imposes the position of its children, then the change will be
undone before the next time around the event loop, so that no
visible effect can be detected.  

If the zone is the root zone and it is connected to some display
server, then depending on the display server, this value may or
may not affect the position of the corresponding top-level window
on the display.

Calling this function triggers the geometry-change protocol,
informing the parent that a change has taken place.  

\subsubsection{Generic function \texttt{set-hpos}}

Set the current horizontal position of the zone relative to its
parent, without triggering the geometry-change protocol.

This function is similar to \texttt{(setf hpos)}, with the difference
that it does not trigger the geometry-change protocol.  It is used
exclusively by internal protocols to avoid infinite recursions when a
parent needs to set the position of a child as a result of some
previous change.

\subsubsection{Generic function \texttt{vpos}}

Return the current vertical position of the zone relative to its
parent.  

If the zone is the root zone and it is connected to some display
server, then depending on the display server, this value may or
may not reflect the position of the corresponding top-level window
on the display.

\subsubsection{Generic function \texttt{(setf vpos)}}

Set the current vertical position of the zone relative to its
parent.

Calling this function is the normal way of setting the position of
a zone that is the child of a layout zone that does not impose the
position of its children, such as a bboard zone or a scroll
zone.

Setting the position of a zone may not have the effect that an
application expects.  If the zone is the child of a layout zone
that imposes the position of its children, then the change will be
undone before the next time around the event loop, so that no
visible effect can be detected.  

If the zone is the root zone and it is connected to some display
server, then depending on the display server, this value may or
may not affect the position of the corresponding top-level window
on the display.

Calling this function triggers the geometry-change protocol,
informing the parent that a change has taken place.  

\subsubsection{Generic function \texttt{set-vpos}}

Set the current vertical position of the zone relative to its
parent, without triggering the geometry-change protocol.

This function is similar to \texttt{(setf vpos)}, with the difference
that it does not trigger the geometry-change protocol.  It is used
exclusively by internal protocols to avoid infinite recursions when a
parent needs to set the position of a child as a result of some
previous change.

\subsubsection{Generic function \texttt{width}}

Return the current width of the zone.  

If the zone is the root zone and it is connected to some display
server, then depending on the display server, this value may or
may not reflect the width of the corresponding top-level window.

\subsubsection{Generic function \texttt{(setf width)}}

Set the current width of the zone.

Calling this function is the normal way of setting the width of a
zone that is the child of a layout zone that does not impose the
size of its children, such as a bboard zone or a scroll zone.

Setting the position of a zone may not have the effect that an
application expects.  If the zone is the child of a layout zone
that imposes the size of its children, then the change will be
undone before the next time around the event loop, so that no
visible effect can be detected.

If the zone is the root zone and it is connected to some display
server, then depending on the display server, this value may or
may not affect the width of the corresponding top-level window
on the display.

Calling this function triggers the geometry-change protocol,
informing the parent that a change has taken place.  

\subsubsection{Generic function \texttt{set-width}}

Set the current width of the zone relative to its parent, without
triggering the geometry-change protocol.

This function is similar to \texttt{(setf width)}, with the difference
that it does not trigger the geometry-change protocol.  It is used
exclusively by internal protocols to avoid infinite recursions when a
parent needs to set the width of a child as a result of some previous
change.

\subsubsection{Generic function \texttt{height}}

Return the current height of the zone.

If the zone is the root zone and it is connected to some display
server, then depending on the display server, this value may or
may not reflect the height of the corresponding top-level window.

\subsubsection{Generic function \texttt{(setf height)}}

Set the current height of the zone.

Calling this function is the normal way of setting the height of a
zone that is the child of a layout zone that does not impose the
size of its children, such as a bboard zone or a scroll zone.

Setting the position of a zone may not have the effect that an
application expects.  If the zone is the child of a layout zone
that imposes the size of its children, then the change will be
undone before the next time around the event loop, so that no
visible effect can be detected.

If the zone is the root zone and it is connected to some display
server, then depending on the display server, this value may or
may not affect the height of the corresponding top-level window on
the display.

Calling this function triggers the geometry-change protocol,
informing the parent that a change has taken place.

\subsubsection{Generic function \texttt{set-height}}

Set the current height of the zone relative to its parent, without
triggering the geometry-change protocol.

This function is similar to \texttt{(setf height)}, with the
difference that it does not trigger the geometry-change protocol.  It
is used exclusively by internal protocols to avoid infinite recursions
when a parent needs to set the height of a child as a result of some
previous change.

\subsubsection{Generic function \texttt{depth}}


\subsubsection{Generic function \texttt{(setf depth)}}

\subsection{Classes}

\subsubsection{Class \texttt{zone}}

The class \texttt{zone} is the base class for all zones. 


\section{Zone genealogy protocol}

\subsection{Generic functions}

\subsubsection{Generic function \texttt{parent}}

Return the current parent of the zone. 

The parent may be another zone, in which case the zone is a child of
that zone, or it may be a client (typically a port), in which case the
zone is the root zone of a hierarchy connected to that client, or it
may be nil, in which case, this zone is the root zone of a hierarchy
not currently connected to any client.

\subsubsection{Generic function \texttt{(setf parent)}}

Set the parent of a zone.

This generic function is part of the internal zone protocols.  It
should not be used directly by applications.  It is called indirectly
as a result of connecting the zone to a client, or as a result of
adding or removing the zone as a child of some other zone by calling
\texttt{(setf children)}.

\subsubsection{Generic function \texttt{children}}

This generic function returns the children of a compound zone.  The
representation of the return value depends on the subclass of
\texttt{compound-zone}.  It could be a list, a vector, a 2-D array, or
something else.

\subsubsection{Generic function \texttt{(setf children)}}

This generic function sets the children of a compound zone.  The
acceptable representation of the children depends on the subclass of
\texttt{compound-zone}.  It could be a list, a vector, a 2-D array, or
something else.  However, the representation is consistent with what
is returned by the \texttt{children} generic function, so that if a list is
returned by that generic function, then a list is acceptable to this
one too.  In particular, it is acceptable to call this function with
the exact same value as was return by a call to \texttt{children}.

There are \texttt{:around} methods on this function (see below).  One
is specialized for \texttt{at-most-one-child-mixin}, and the other for
\texttt{any-number-of-children-mixin}, both subclasses of
\texttt{compound-zone}.  These \texttt{:around} methods always call
the primary methods, but they also do some extra work such as error
checking, setting the parent of every new child, removing the parent
of every removed child, and client notification.  The \texttt{:around}
method calls \texttt{children}, which has as a consequence that in
order to use \texttt{(setf children)} the corresponding slot must be
bound.

There is one \texttt{:after} method specialized for
\texttt{compound-zone} that calls

\subsubsection{generic function \texttt{map-over-children}}

the first argument is a function of a single argument.  the second
argument is a zone.  this function calls the function given as the
first argument on each child of the zone given as a second argument.  

\subsubsection{generic function \texttt{map-over-children-top-to-bottom}}

the first argument is a function of a single argument.  the second
argument is a zone.  this function calls the function given as the
first argument on each child of the zone given as a second argument.  

\subsubsection{generic function \texttt{map-over-children-bottom-to-top}}

The first argument is a function of a single argument.  The second
argument is a zone.  This function calls the function given as the
first argument on each child of the zone given as a second argument.  

\subsection{Classes}

\subsubsection{Class \texttt{atomic-zone}}

\subsubsection{Class \texttt{compound-zone}}

\subsubsection{Class \texttt{compound-simple-zone}}

\subsubsection{Class \texttt{compound-sequence-zone}}

\subsubsection{Class \texttt{compound-matrix-zone}}

\subsubsection{Class \texttt{at-most-one-child-mixin}}

\subsubsection{Class \texttt{any-number-of-children-mixin}}


\section{Sprawls change protocol}

FIXME: this section no longer reflects recent changes, so it needs
updating.

The sprawls change protocol is divided into four functions for maximum
flexibility to client code, three of them generic, and one is an
ordinary function.

\subsection{Generic functions and methods}

\subsubsection{Generic function \texttt{sprawls-valid-p}}
\label{generic-function-sprawls-valid-p}

This function returns true if and only if the sprawls of the zone are
valid.  

For zones with sprawls that can not be invalidated, such as bboard
zones, scroll zones, and atomic zones with sprawls that do not depend
on a client, this function always returns \emph{true}.

The methods of this function are paired up with methods for
\texttt{mark-sprawls-invalid} so that they both use the same mechanism. 

This function should return \emph{false} after a call has been made to
\texttt{mark-sprawls-invalid}, and should return \emph{true} after a
call has been made to \texttt{compute-sprawls}.  

A method specialized for \texttt{dependent-sprawls-mixin} is supplied
that returns \emph{true} of and only if both the \texttt{vsprawl} and
the \texttt{hsprawl} of the zone are \texttt{nil}

A method specialized for \texttt{hdependent-sprawls-mixin} is supplied
that returns \emph{true} if and only if the \texttt{hsprawl} of the zone
is \texttt{nil}.

A method specialized for \texttt{vdependent-sprawls-mixin} is supplied
that returns \emph{true} if and only if the \texttt{vsprawl} of the zone
is \texttt{nil}.

A method specialized for \texttt{independent-sprawls-mixin} is supplied
that always returns \emph{true}. 

\subsubsection{Generic function \texttt{mark-sprawls-invalid}}
\label{generic-function-mark-sprawls-invalid}

This function marks the zone as having invalid sprawls.  

For zones with sprawls that can not be invalidated, such as bboard
zones, scroll zones, and atomic zones with sprawls that do not depend
on a client, this function signals an error.

Different subclasses of \texttt{zone} do this differently.  Some set
both the hsprawl and the vsprawl to \texttt{nil}.  Some others set one but
not the other.  Some may use a completely different mechanism.

The methods of this function are paired up with methods for
\texttt{sprawls-invalid-p} so that they both use the same mechanism.

A method specialized for \texttt{dependent-sprawls-mixin} is supplied
that sets both the \texttt{vsprawl} and the \texttt{hsprawl} of the zone
to \texttt{nil}

A method specialized for \texttt{hdependent-sprawls-mixin} is supplied
that sets the \texttt{hsprawl} of the zone to \texttt{nil}.

A method specialized for \texttt{vdependent-sprawls-mixin} is supplied
that sets the \texttt{vsprawl} of the zone to \texttt{nil}.

A method specialized for \texttt{independent-sprawls-mixin} is supplied
that signals an error.

\subsubsection{Generic function \texttt{notify-child-sprawls-invalid}}
\label{generic-function-notify-child-sprawls-invalid}

This function is called on a child and a parent when the sprawls of the
child are invalidated for some reason (see the function
\texttt{invalidate-sprawls}).  Methods on this generic function
specialized to parent zones (the second argument) whose sprawls depend
on the sprawls of its children must call \texttt{invalidate-sprawls} with
the parent as an argument.

Two default methods are supplied.  The first default method is
specialized on a \texttt{null} parent and it does nothing.  This way,
a zone and its parent are always valid arguments to this function,
even though the zone is disconnected.  The second default method is
specialized on a parent of type \texttt{zone} and it signals an error.
By doing it this way, we compel zones to make an explicit choice
concerning the action of this function.

Other methods are supplied for certain mixin classes.  For the mixin
classes \texttt{dependent-sprawls-mixin},
\texttt{hdependent-sprawls-mixin}, and \texttt{vdependent-sprawls-mixin},
a method is supplied that calls \texttt{invalidate-sprawls} with the
parent as an argument, as stipulated above.  For the class
\texttt{independent-sprawls-mixin}, a method is supplied that does
nothing.

\subsubsection{Function \texttt{invalidate-sprawls}}

This function is used to indicate that the sprawls of some zone are no
longer valid.  The reason for that could be that the sprawls depend on
the contents and the contents changed, or that the sprawls of the zone
depend on it being connected to some client, and it got disconnected.

If the sprawls of the zone are already invalid, i.e.,
\texttt{sprawls-valid-p} returns \emph{false}, then this function does
nothing.  Otherwise, it marks them invalid by calling
\texttt{mark-sprawls-invalid} and calls the function
\texttt{notify-child-sprawls-invalid} with the zone and its parent.

\section{Zone layout protocol}

The zone layout protocol is designed so that if the zone hierarchy is
appropriately designed, then layout can be done incrementally for good
performance. 

In any hierarchy of zones, some zones may have valid sprawls and some
zones may have invalid sprawls.  Atomic zones whose sprawls do not depend
on any client are always valid.  Atomic zones whose sprawls depend on
some client may or may not be valid.  If the zone hierarchy is not
connected to a client, they are invalid. 

A compound zone whose sprawls depend on the sprawls of its children will
have invalid sprawls if any of the children has invalid sprawls.  We
maintain this invariant by using the sprawls change protocol.  Thus, the
only case where a (compound) zone can have valid sprawls even though
some child has invalid sprawls is when that zone is such that its sprawls
are independent of the sprawls of the children.  In particular, this is
the case for scroll zones and bboard zones.

When a zone is rendered, its position and size must be known.  In
addition, we require that a zone that is rendered also have valid
sprawls, even though this is not strictly necessary.

The size and sometimes the position of a zone are set by the parent.
If the parent is a port, then the port must set these parameters
before rendering the zone.  Sometimes, when a zone is rendered, it is
possible that its children do not have the correct position and size.
For instance, if the position of a child of a hbox zone is modified by
the application, then it is likely wrong, so it must be fixed before
the hbox zone can be rendered.  Similarly, if the sprawls of a child of
a bbox zone have changed, then the natural size of that child is very
likely wrong, so it must be fixed before the bbox zone can be
rendered.

When the position or the sprawls of a zone change, we want to delay the
fixes until the parent zone must be rendered.  We do this by
maintaining a flag in compound zones that indicate whether the child
geometries are valid or not.  

\subsection{Generic functions}

\subsection{Classes}

\subsubsection{Class \texttt{dependent-sprawls-mixin}}
\subsubsection{Class \texttt{hdependent-sprawls-mixin}}
\subsubsection{Class \texttt{vdependent-sprawls-mixin}}
\subsubsection{Class \texttt{independent-sprawls-mixin}}

\section{Application frame}

An application frame is a layout zone with a small number of fixed
layouts of sub-zones (we don't call them child zones because they
are not the direct children of the application frame.  Selecting a
particular layout has the effect of creating child zones of the
application frame with positions and sizes that are determined by the
layout, and each child contains a sub-zone.

\section{Table zone}

\section{Simple application zone}

This is a zone that executes a display function each time around the
command loop in order to generate a complete hierarchy of zones.

\section{Incremental application zone}

This is a zone that is able to compare existing child zones with
new ones, and that will recursively generate zones only if a child
is new. 

